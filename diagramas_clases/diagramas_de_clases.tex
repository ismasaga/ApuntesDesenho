\documentclass[10pt,a4paper]{article}
\usepackage[latin1]{inputenc}
%define idioma
\usepackage[spanish]{babel}
\usepackage{amsmath}
\usepackage{amsfonts}
\usepackage{amssymb}
\usepackage{graphicx}
% define el t�tulo
\title{Diagramas de clases}
\author{Ismael Saborido}
\begin{document}
	% genera el t�tulo
	\maketitle
	% inserta el �ndice general
	\tableofcontents
	
	\section{Definici�n}
	Unha \textbf{clase} � unha descrici�n dun conxunto de obxectos que comparten atributos, operaci�ns,relaci�ns e sem�ntica.\\
	Son abstracci�ns independentes da linguaxe de programaci�n capaces de implementar interfaces. Os obxectos(equivalente a instancias) da mesma clase te�en os mesmos tipos de estado e comportamento.\\
	As clases colaboran entre si para satisfacer necesidades. Hai 3 tipos de relaci�ns: 
	\begin{itemize}
		\item dependencia (os cambios na especificaci�n dun elemento poden afectar a outro, indica utilizaci�n)
		\item xeralizaci�n (relaci�n pai/fillo)
		\item asociaci�n (relaci�n estrutural)
	\end{itemize}
	Son a base para diagramas de paquetes, compo�entes e despregue. Poden conter paquetes.
	
	
	\begin{thebibliography}{X}
		%\begin{itemize}
			Na elaboraci�n deste documento foi empregado un recurso en li�a que se pode atopar en www.gradox.org 
			titulado \textit{apuntes.pdf} cuxa autor�a pertence a \textsc{Cristina S. Barreiro} visionado por �ltima vez o \date{23 de Febrero de 2016}.
		%\end{itemize}
	\end{thebibliography}
	
	
%EJEMPLO BIBLIOGRAF�A	
%	\begin{thebibliography}{X}
%		\bibitem{Baz} \textsc{Bazaraa, M.S., J.J. Jarvis} y \text
%		sc{H.D. Sherali},
%		\textit{Programaci�n lineal y flujo en redes}, segunda e
%		dici�n,
%		Limusa, M�xico, DF, 2004.
%		\bibitem{Dan} \textsc{Dantzig, G.B.} y \textsc{P. Wolfe}
%		,
%		<<Decomposition principle for linear programs>>,
%		\textit{Operations Research}, \textbf{8}, p�gs. 101--11
%		1, 1960.
%	\end{thebibliography}
	
\end{document}